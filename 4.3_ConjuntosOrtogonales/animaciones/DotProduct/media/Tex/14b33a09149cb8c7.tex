\documentclass[preview]{standalone}

\usepackage[english]{babel}
\usepackage[utf8]{inputenc}
\usepackage[T1]{fontenc}
\usepackage{lmodern}
\usepackage{amsmath}
\usepackage{amssymb}
\usepackage{dsfont}
\usepackage{setspace}
\usepackage{tipa}
\usepackage{relsize}
\usepackage{textcomp}
\usepackage{mathrsfs}
\usepackage{calligra}
\usepackage{wasysym}
\usepackage{ragged2e}
\usepackage{physics}
\usepackage{xcolor}
\usepackage{microtype}
\DisableLigatures{encoding = *, family = * }
\linespread{1}

\begin{document}

\begin{center}
\quad\\justifyingSea \textit{V} un espacio vectorial con producto interno y sea \quad\\\ \quad\\\ $B = \{\vec{v_1}, \vec{v_2}, ..., \vec{v_n}\}$ \quad\\\ \quad\\\ una base ortogonal de \textit{V}. Si $\vec{a} \in V$ y se tiene que:\quad\\\ \quad\\\ $\vec{a} = \{\alpha_1 \vec{v_1} + \alpha_2 \vec{v_2} + ... + \alpha_n \vec{v_n} \}$ \quad\\\ \quad\\\ entonces los escalares $\alpha_i$ vienen dados por la expresión: \quad\\\ \quad\\\ $\alpha_i = \frac{(\vec{a} | \vec{v_i})}{(\vec{v_i} | \vec{v_i})} $
\end{center}

\end{document}
