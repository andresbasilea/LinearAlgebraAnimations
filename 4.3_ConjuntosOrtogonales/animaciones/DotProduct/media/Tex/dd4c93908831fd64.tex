\documentclass[preview]{standalone}

\usepackage[english]{babel}
\usepackage[utf8]{inputenc}
\usepackage[T1]{fontenc}
\usepackage{lmodern}
\usepackage{amsmath}
\usepackage{amssymb}
\usepackage{dsfont}
\usepackage{setspace}
\usepackage{tipa}
\usepackage{relsize}
\usepackage{textcomp}
\usepackage{mathrsfs}
\usepackage{calligra}
\usepackage{wasysym}
\usepackage{ragged2e}
\usepackage{physics}
\usepackage{xcolor}
\usepackage{microtype}
\DisableLigatures{encoding = *, family = * }
\linespread{1}

\begin{document}

\begin{center}
\quad\\justifying {Sea \textit{V} un espacio vectorial sobre un campo de definición complejo. Un \textbf{producto interno} es una función de \textit{VxV} en $\mathbb{C}$ que asocia a cada pareja de vectores $\vec{u}$ y $\vec{v}$ de \textit{V} un escalar $(\vec{u} | \vec{v})$ $\in \mathbb{C}$, llamado el producto interno de  $\vec{u}$ y $\vec{v}$.}
\end{center}

\end{document}
