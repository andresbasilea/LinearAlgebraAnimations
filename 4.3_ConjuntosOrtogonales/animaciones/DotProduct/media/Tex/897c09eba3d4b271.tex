\documentclass[preview]{standalone}

\usepackage[english]{babel}
\usepackage[utf8]{inputenc}
\usepackage[T1]{fontenc}
\usepackage{lmodern}
\usepackage{amsmath}
\usepackage{amssymb}
\usepackage{dsfont}
\usepackage{setspace}
\usepackage{tipa}
\usepackage{relsize}
\usepackage{textcomp}
\usepackage{mathrsfs}
\usepackage{calligra}
\usepackage{wasysym}
\usepackage{ragged2e}
\usepackage{physics}
\usepackage{xcolor}
\usepackage{microtype}
\DisableLigatures{encoding = *, family = * }
\linespread{1}

\begin{document}

\begin{center}
\justifying {Primero obtendremos un generador ortogonal de dicho espacio, mediante el proceso de Gram-Schmidt} \\\  Para ello hacemos: \\\  $\vec{w_1} = \vec{v_1} = (1, 0, -1)$  \newline $\vec{w_2} = \vec{v_2} - \frac{(\vec{v_2} | \vec{w_1})}{(\vec{w_1} | \vec{w_1})} \vec{w_1} = (-2, 1, 1) - \frac{-3}{2} (1, 0, -1) = (\frac{-1}{2}, 1, \frac{-1}{2} ) $ \newline $\vec{w_3} = \vec{v_3} - \frac{(\vec{v_3} | \vec{w_1})}{(\vec{w_1} | \vec{w_1})} \vec{w_1} - \frac{(\vec{v_3} | \vec{w_2})}{(\vec{w_2} | \vec{w_2})} \vec{w_2} $\newline $ = (-1, 1, 0) - \frac{-1}{2} (1, 0, -1) - \frac{\frac{-3}{2}}{\frac{-3}{2}} (\frac{-1}{2}, 1, \frac{-1}{2}) = (0, 0, 0) $\justifying {entonces: $G_0 = \{ (-1, 0, -1), (\frac{-1}{2}, 1, \frac{-1}{2}), (0, 0, 0) \}$} \\\ \\\ \justifying {es un generador ortogonal de \textit{V} y el conjunto}$B = \{ (1, 0, -1), (\frac{-1}{2}, 1, \frac{-1}{2}) \} $ es una base ortogonal de dicho espacio.
\end{center}

\end{document}
