\documentclass[preview]{standalone}

\usepackage[english]{babel}
\usepackage[utf8]{inputenc}
\usepackage[T1]{fontenc}
\usepackage{lmodern}
\usepackage{amsmath}
\usepackage{amssymb}
\usepackage{dsfont}
\usepackage{setspace}
\usepackage{tipa}
\usepackage{relsize}
\usepackage{textcomp}
\usepackage{mathrsfs}
\usepackage{calligra}
\usepackage{wasysym}
\usepackage{ragged2e}
\usepackage{physics}
\usepackage{xcolor}
\usepackage{microtype}
\DisableLigatures{encoding = *, family = * }
\linespread{1}

\begin{document}

\begin{center}
\justifying {Obtener una base ortonormal del espacio V generado por los vectores: } \\\ \\\ \begin{center} {$\vec{v_1} = (1, 0, -1)$ } \end{center} \\\ \begin{center} {$\vec{v_2} = (-2, 1, 1)$ } \end{center} \\\ \begin{center} {$\vec{v_1} = (-1, 1, 0)$ } \end{center} \\\ \justifying {con las operaciones usuales de adición, multiplicación por un escalar y producto interno en R^3} \\\ \\\
\end{center}

\end{document}
