\documentclass[preview]{standalone}

\usepackage[english]{babel}
\usepackage[utf8]{inputenc}
\usepackage[T1]{fontenc}
\usepackage{lmodern}
\usepackage{amsmath}
\usepackage{amssymb}
\usepackage{dsfont}
\usepackage{setspace}
\usepackage{tipa}
\usepackage{relsize}
\usepackage{textcomp}
\usepackage{mathrsfs}
\usepackage{calligra}
\usepackage{wasysym}
\usepackage{ragged2e}
\usepackage{physics}
\usepackage{xcolor}
\usepackage{microtype}
\DisableLigatures{encoding = *, family = * }
\linespread{1}

\begin{document}

\begin{center}
\justifying {Para obtener una base ortonormal calculamos:} \\\  $||\vec{w_1}|| = \sqrt{(\vec{w_1}|\vec{w_1})} = \sqrt{2}$ \\\  $||\vec{w_2}|| = \sqrt{(\vec{w_2}|\vec{w_2})} = \sqrt{\frac{3}{2}} $ \\\  y en consecuencia el conjunto: \\\  $B = \{ (\frac{1}{\sqrt{2}}, 0, \frac{-1}{\sqrt{2}}), (\frac{-1}{\sqrt{6}}, \sqrt{\frac{2}{3}}, \frac{-1}{\sqrt{6}}) \} $ \\\  es una base ortonormal de V
\end{center}

\end{document}
